\documentclass{article}
\usepackage{amsmath, amssymb}
\usepackage{amsthm}
\usepackage{geometry}
\usepackage{hyperref}  % last or near last
\usepackage{longtable}
\usepackage{array}
\usepackage{xeCJK}            % 中文支持
\usepackage{fontspec}         % 字体设置(可选)
%\setCJKmainfont{SimSun} 

\geometry{margin=1in}
\title{Understanding and Attempting Proof of Goldbach Conjecture}
\author{Zhikai Wang zhikai.wang@alumni.ucla.edu}
\date{June 2025}

\begin{document}

\maketitle

\section{Introduction}
In this article, we attempt to proove or provide a better understanding of this well known problem. Here, we also ignore obvious mathematical terms and facts. Most computer science and mathematics graduates will have sufficient background to understand the content. We also ignore very edge cases and focus on the way of thinking. We also do not attempt to proof obvious facts.
\section{Definitions}
\subsection*{N-multiple}
An integer has n prime repeatable integer factors. $n \in \mathbb{N} \setminus {1}$. Examples, $2\times 3$,$2\times 2\times 3$ and $3\times 5\times7$.
\subsection*{Odd Multiple}
We only use nature numbers as the factors, in this case, $1$ and $2$ are removed.
\section{Axioms}

\noindent \textbf{Axiom (Prime number):} \\
Prime numbers are the holes in odd numbers such that odd multiples fail to fill in. It can also be understood as nature numbers subtract even numbers and odd multiples.
\bigskip

\noindent \textbf{Axiom (Neighor Numbers are Co-prime):} \\
Or 邻数因子互斥 in Chinese. Given any number $n$, $n$ and $n+1$ are co-prime. If we put $n$ in unique factorial format eg
$p_1^{e_1} \cdot p_2^{e_2} \cdots p_l^{e_l}$, it will be more obvious $p_1^{e_1} \cdot p_2^{e_2} \cdots p_l^{e_l} + 1$ does not share any factor with $n$. Further $n-1$ does not share any factor with $n$.
\bigskip

\noindent \textbf{Axiom (Neighboring Zone Factorization Analysis):} \\
For number sequence $n$, $n-1$, $n-i$, $\cdots $, $n-m$, $m < n$. We call $i$ the $\delta $. In this zone, each smaller number only share factors with larger numbers that are factors of current $\delta $ and previous $\delta $s. When $\delta $ is prime and larger than any factors from applicable $\delta $s, it will force in a prime that is never used in larger numbers' factorization.
\bigskip

\newpage
\section{Application in Goldbach Conjecture}
Let's try with $85292$ to show how to calculate the Goldbach prime pair $79$ and $85213$, partly manual with computer program aid. I also use Chatgpt aiding Latex writing. Pay attention to steps at $m=3,5,7,11,13,17,19,23,29,31,37,41,47,53,67,71,73,79$.

\begin{proof}[Proof of Goldbach Conjecture]
In zone $[n-m,n]$, each time a new prime never used is forced in when $\delta$ is a prime. Combining the distinct primes used in even numbers, the primes less equal to $\sqrt n$ soon run out. We can only fill in one but ONLY ONE prime that is bigger than $\sqrt n$.
\end{proof}


\renewcommand{\arraystretch}{1.2}
\begin{longtable}{|>{\centering\arraybackslash}p{2.5cm}|>{\arraybackslash}p{6.5cm}|>{\centering\arraybackslash}p{1.5cm}|}
\caption{n=85292}
\label{tab:85292} 
\hline
\textbf{$n - m$} & \textbf{Factorization of $n - m$} & \textbf{$-m$} \\
\hline
\endfirsthead
\hline
\textbf{$n - m$} & \textbf{Factorization of $n - m$} & \textbf{$-m$} \\
\hline
\endhead
85212 & 2×2×3×3×3×3×263 & -80 \\
85213 & 85213 & -79 \\
85214 & 2×137×311 & -78 \\
85215 & 3×5×13×19×23 & -77 \\
85216 & 2×2×2×2×2×2663 & -76 \\
85217 & 11×61×127 & -75 \\
85218 & 2×3×7×2029 & -74 \\
85219 & 31×2749 & -73 \\
85220 & 2×2×5×4261 & -72 \\
85221 & 3×3×17×557 & -71 \\
85222 & 2×42611 & -70 \\
85223 & 85223 & -69 \\
85224 & 2×2×2×3×53×67 & -68 \\
85225 & 5×5×7×487 & -67 \\
85226 & 2×43×991 & -66 \\
85227 & 3×28409 & -65 \\
85228 & 2×2×11×13×149 & -64 \\
85229 & 85229 & -63 \\
85230 & 2×3×3×5×947 & -62 \\
85231 & 29×2939 & -61 \\
85232 & 2×2×2×2×7×761 & -60 \\
85233 & 3×28411 & -59 \\
85234 & 2×19×2243 & -58 \\
85235 & 5×17047 & -57 \\
85236 & 2×2×3×7103 & -56 \\
85237 & 85237 & -55 \\
85238 & 2×17×23×109 & -54 \\
85239 & 3×3×3×7×11×41 & -53 \\
85240 & 2×2×2×5×2131 & -52 \\
85241 & 13×79×83 & -51 \\
85242 & 2×3×14207 & -50 \\
85243 & 85243 & -49 \\
85244 & 2×2×101×211 & -48 \\
85245 & 3×5×5683 & -47 \\
85246 & 2×7×6089 & -46 \\
85247 & 85247 & -45 \\
85248 & 2×2×2×2×2×2×2×2×3×3×37 & -44 \\
85249 & 163×523 & -43 \\
85250 & 2×5×5×5×11×31 & -42 \\
85251 & 3×157×181 & -41 \\
85252 & 2×2×21313 & -40 \\
85253 & 7×19×641 & -39 \\
85254 & 2×3×13×1093 & -38 \\
85255 & 5×17×17×59 & -37 \\
85256 & 2×2×2×10657 & -36 \\
85257 & 3×3×9473 & -35 \\
85258 & 2×47×907 & -34 \\
85259 & 85259 & -33 \\
85260 & 2×2×3×5×7×7×29 & -32 \\
85261 & 11×23×337 & -31 \\
85262 & 2×89×479 & -30 \\
85263 & 3×97×293 & -29 \\
85264 & 2×2×2×2×73×73 & -28 \\
85265 & 5×17053 & -27 \\
85266 & 2×3×3×3×1579 & -26 \\
85267 & 7×13×937 & -25 \\
85268 & 2×2×21317 & -24 \\
85269 & 3×43×661 & -23 \\
85270 & 2×5×8527 & -22 \\
85271 & 71×1201 & -21 \\
85272 & 2×2×2×3×11×17×19 & -20 \\
85273 & 269×317 & -19 \\
85274 & 2×7×6091 & -18 \\
85275 & 3×3×5×5×379 & -17 \\
85276 & 2×2×21319 & -16 \\
85277 & 53×1609 & -15 \\
85278 & 2×3×61×233 & -14 \\
85279 & 107×797 & -13 \\
85280 & 2×2×2×2×2×5×13×41 & -12 \\
85281 & 3×7×31×131 & -11 \\
85282 & 2×42641 & -10 \\
85283 & 11×7753 & -9 \\
85284 & 2×2×3×3×23×103 & -8 \\
85285 & 5×37×461 & -7 \\
85286 & 2×42643 & -6 \\
85287 & 3×28429 & -5 \\
85288 & 2×2×2×7×1523 & -4 \\
85289 & 17×29×173 & -3 \\
85290 & 2×3×5×2843 & -2 \\
85291 & 19×67×67 & -1 \\
85292 & 2×2×21323 & 0 \\
\hline
\end{longtable}

\newpage


\section{Final Words}
It is still a little bit surreal for me to get here. I have sufficient background in computer science and mathematics in understanding what I am doing. But Number Theory is not my specialty and particularly I aimed too high for this conjecture. I hope the readers please do not be afraid to criticize any mistakes. I hope if the conjecture is not rigidly proved here, at least my article may provide any meaningful clues for further research. Though this is not my first academic writing, still these "final words" may serve an "excuse me" if there is any grieve mistake.
Solute the all the people dedicate to research in computer science and mathematics.
\end{document}

